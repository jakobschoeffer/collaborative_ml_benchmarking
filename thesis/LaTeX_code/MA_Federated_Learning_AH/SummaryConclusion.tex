\section{Summary and conclusion\label{sec:summary}}
Motivated by the Service-Meister research project and the need for approaches to overcome the knowledge and data issues SMEs face when aiming at applying ML, this thesis investigates FL from an SME perspective, analyzing \emph{performance}, \emph{privacy}, and \emph{complexity} consequences and implications for SMEs in the federation.

From a \emph{performance} perspective, we find that the more challenging an SME's data situation is (the fewer data an SME has and the more unbalanced the label distribution of an SME's data is in comparison to the other SMEs), the more the SME benefits from FL in comparison to training an individual model solely with its own resources.
The performance gains of SMEs, measured by the AUC increase that can be achieved in the FL setting compared to the \emph{one model per client} setting, are positive in all our simulations, meaning that there is no disadvantage resulting from taking part in FL even for SMEs with a strong individual data situation, though their incentive might be limited. There is a clear incentive towards taking part in FL from a performance perspective.

FL aims to ensure \emph{privacy}, but there are potential privacy threats that have to be kept in mind by SMEs. We find, that currently only the server-side mGAN-AI \citep{wang2019mGANAI} attack is a realistic threat but can be prevented by ensuring that the server is trustworthy. The other attacks listed in \citet{enthoven2021overview} are either unrealistic in practice or can be avoided using relatively simple but powerful defense measures such as dropout or artificial noise.

Concerning \emph{complexity}, we find that in contrast to in classical FL use cases \emph{computational complexity} is not a dominating factor for SMEs. The \emph{implementation complexity} for each SME in FL can be expected to be lower than in the \emph{one model per client} setting due to the potential to share the modeling efforts and only limited additional complexity resulting from establishing the federation from a technical perspective, e.g., connecting to the server and providing a client instance for training. We find \emph{organizational complexity}, the complexity that arises from organizing and facilitating FL, to be the key driver of complexity in the SME context: for example, SMEs with related use cases have to be identified, approached, and potentially convinced, contracts have to be negotiated, and a cost-sharing model has to be established. Consequently, this leads to a considerable non-technical additional complexity.

Finally, we find a clear need for a cost-sharing model that guarantees adequate incentives in the federation. As SMEs with the most challenging data situation profit the most from FL, a cost-sharing model that splits costs equally among the federating would yield an incentive structure that promotes free-riding in the sense of putting as little effort as possible in contributing high-quality and high-quantity data. The performance gains can be seen as a first step towards measuring how much each SME profits from FL. Further research concerning organizational frameworks and cost-sharing models can contribute to simplifying the application of FL in the SME context.

In summary, FL offers a considerable opportunity for SMEs to address their problems of limited resources and knowledge to remain competitive in the future. If privacy requirements are met, and the performance advantages outweigh the additional complexity, FL is a great chance for SMEs to pool data and knowledge while preserving privacy.
