{\Large \textbf{Abstract}}

\bigskip

Small and medium-sized enterprises (SMEs), which play a vital role in the economy, especially in Germany, increasingly want to make use of machine learning (ML), but face two major challenges: the lack of knowledge and the lack of sufficient amounts of data for training adequate models. In this work, we empirically show that federated machine learning (FL) can help SMEs with related ML use cases overcome these problems by making use of the joint data of all SMEs taking part in the federation while preserving privacy. We identify and analyze three dimensions regarding which we assess the effectiveness of FL in the SME context: \emph{performance}, \emph{privacy}, and \emph{complexity}. For the \emph{performance} dimension, we provide a simulation pipeline that lets us simulate realistic FL scenarios. We find that, regarding \emph{performance}, all SMEs potentially profit from taking part in the federation and that SMEs with a particularly challenging data situation tend to profit the most. Regarding \emph{privacy}, as only model weights are exchanged, we note that privacy requirements are usually fulfilled -- despite there being potential privacy attacks. We argue that these attacks are limited in power in common SME FL contexts and might be avoidable using basic defense strategies. Regarding \emph{complexity}, we argue that implementation and organizational complexity are crucial in the SME context, as opposed to computational complexity, which is more critical in other use cases.
